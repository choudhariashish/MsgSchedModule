\section{Background and Related Work}
\label{sec:related_work}

CPSs, with few exceptions, are switched dynamic systems. A switched system is a
fundamentally continuous-time system with changes that occur at discrete
times~\cite{liberzon03}. Analysis and design of CPSs is a challenge as any
process must simultaneously take into account cyber, physical, and network
aspects. Some work is breaking through this barrier.
Acumen~\cite{Zhu:2010:MEE:1795194.1795196} bridges the gap between analytic
models and simulation codes. Interface automata~\cite{Alfaro} checks for
compatibility of components in composition by showing that they do not
\emph{interfere} with each other. Recent work~\cite{DAC_2012_CPS_verification}
proposes a performance verification technique for CPS. This work assumes the use
of a communication network such as CAN, FlexRay, etc. and use the relatively
structured properties of the networking infrastructure to tightly bound network
delays and, hence, control performance. Invariants and predicate transformers on
the state of CPS was explored for dynamical systems in~\cite{DBLP:SitzoffG93}
and more recently as a formalism for invariant interaction and incremental
invariant composition~\cite{5587717} and run-time assurance of operational
modes~\cite{Bak2010}. The interaction of invariants for purely cyber processes
has its origins in~\cite{OwickiGries1976} which affords composition of
sequential proofs governed by the property of noninterference.
Recently, there have been several attempts to create comprehensive models for
design and analysis of CPSs (such
as~\cite{DerlerLeeSangiovanniVincentelli12_ModelingCyberPhysicalSystems,
GarlanKrogh2010,cyphy}) that use domain-specific ontologies and hybrid systems
techniques.

Correct scheduling of actions affecting one or more sub-components is key in
such a CPS in order to maintain overall system stability. However, stability and
scheduling are not {\em a-priori}, but must be adaptive based on events in the
CPS. Mode-based real-time scheduling allows different modes of operation where
different modes may have variation in their task set and/or task timing
characteristics~\cite{sha_rts_1989, real_crespo_rts_2004, henia_ernst_rtss_2007,
stoimenov_date_2009}, thereby allowing a degree of adaptation. However, existing
approaches assume that mode parameters and mode change triggers are statically
well-defined, allowing static analysis of individual modes and mode transitions,
thus making them inapplicable in a CPS.
Recent work proposes a technique for online reconfiguration of resource
reservations using Constant Bandwidth
Servers~\cite{ReconfigurationCBSECRTST2012}. Elastic scheduling
strategies~\cite{Buttazzo_IEEE_Jornl_2002} and feedback
schedulers~\cite{FeedbackScheduling07} allow for more dynamic adaptation, but
adaptation is still typically performed at sporadic intervals, in contrast to
the continuous adaptation needed in a CPS. Adaptive scheduling as
in~\cite{AdaptiveScheduling_DASC_1999} dynamically changes the rates of task
execution in response to system behavior, but would require complete abstraction
of physical and network parameters, making its application to a CPS very
challenging. Scheduling of power demands for optimal energy management in a
smart grid has been proposed~\cite{SmartGridLoadScheduling}. However, this work
only considers instantaneous power in the physical system and does not consider
network behavior. Considering all continuous and discrete dynamics along with
dynamic behavior simultaneously results in state explosion. While verification
is possible, it is extremely challenging.

[******add citations]
Although protocols such as TCP are well known for reliability, TCP relies on packet 
drops caused due to overflow of queues at routers as an indication of network congestion. 
In smart grid context, every message is responsible for a small amount of power in
grid therefore message loss is directly associated to the grid stability. TCP by controlling 
its transmission rate can effectively reduce packet drops if transport is capable of ECN.
But, congestion information obtained from network is completely hidden from the application
running over TCP. In order to know ECN in application, UDP protocol is a perfect choice.
It is then application's responsibility to adapt according to network conditions and take 
necessary actions upon detection of congestion in network.
[******end citations]

