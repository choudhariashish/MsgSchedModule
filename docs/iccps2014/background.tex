\section{Background and Related Work}
\label{sec:related_work}

Most of the Cyber-physical systems are switched, continuous time dynamic systems with changes occurring at discrete time intervals~\cite{liberzon03}. Analyzing CPS is a complicated task, as it consists of tight coupling of various components that are heterogeneous in nature, such as, cyber,  physical and network. In CPS, actions performed in one component should not affect the stability in another components. Therefore, it is mandatory to schedule \textit{correct} actions at \textit{correct} time in each component of CPS in order to maintain overall system stability. However, only stability and correct scheduling is not enough, the system should also be adaptive in nature when uncertain components are present in the composed CPS. Typically, most of the CPS kind of applications, involve communication network as one of the important components ("smart grid" as a prime example). Standalone physical systems with computational capability, such as power plants, vehicular systems, medical devices, and robotics - just to name few - have been deeply studied in the literature, but when these kind of systems are interconnected through unreliable or uncertain communication network, like internet, physical system aspects, assumptions, control strategies and functional behavior is affected due to the unpredictability of transmission time in the communication network. 

A verification technique for Cyber-Physical systems was recently proposed in~\cite{DAC_2012_CPS_verification} assuming communication network to be CAN, FlexRay. etc, wherein authors use structured properties of network infrastructures and efficiently bound network delays. 
In~\cite{SmartGridLoadScheduling}, authors propose an idea of scheduling power demands for optimal energy management in smart-grid, but they do not consider the network behavior. In traditional real-time systems, dynamic adaptation is normally performed at "sporadic intervals" using elastic scheduling techniques~\cite{Buttazzo_IEEE_Jornl_2002} and feedback scheduling~\cite{FeedbackScheduling07}, but CPS requires continuous adaptation. Adaptive scheduling proposed in ~\cite{AdaptiveScheduling_DASC_1999}, dynamically changes the task execution rate based on the observed system behavior, but requires complete abstraction of physical and network parameters. Similar approach can be taken to dynamically change the power transfer rate across the smart-grid nodes based on the observed network conditions, but abstraction of internet like network parameters (eg. RTT, Round-Trip time) is non-trivial. In our recent paper~\cite{acsmartgrid}, we proposed an adaptive algorithm which schedules power migrate messages between CPS smart-grid nodes based on the observed round-trip times in internet like network. This algorithm forms invariant which in conjunction with physical system invariant achieves stability of the overall system. The main reason of unpredictable round-trip times (in internet) is the nature of the traffic, increase in traffic might lead to exponential increase in transmission delays and can also cause messages to drop.

[******add citations]
{\bf Protocols for CPS,} although internet protocols such as TCP are well known for reliability, TCP relies on packet drops caused due to overflow of queues at gateways as an indication of network congestion. 
In smart grid context, every message is responsible for a small amount of power in the
grid, therefore a loss of message is directly associated to the grid stability. TCP by controlling 
its transmission rate can effectively reduce packet drops if transport is capable of ECN.
But, congestion information obtained from network is completely hidden from the application
running over TCP. In order to know ECN in application, User Datagram Protocol (UDP) is a perfect choice.
It is then application's responsibility to adapt according to network conditions and take 
necessary actions upon detection of congestion in the network.
[******end citations]

